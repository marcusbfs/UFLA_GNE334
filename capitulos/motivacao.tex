\chapter{Motivação}\label{motivacao}
Para o desenvolvimento de um projeto industrial que envolva transporte de 
fluidos é necessário o conhecimento do
estado energético-mecânico do fluido para que o mesmo possa ser deslocado 
conforme a necessidade do sistema.  Avaliando,
neste caso, a necessidade ou não da utilização de sistemas fluidomecânicos 
(bombas, compressores, por exemplo). 

A aferição do estado energético-mecânico do fluido em um determinado ponto de 
uma tubulação, por exemplo, requer a
investigação precisa dos pontos na tubulação por onde passou e o estado 
inicial do mesmo. Desta forma, elencar-se-á as
perdas de carga envolvidas durante o processo e, poder-se-á, então, determinar 
o estado de um fluido em um ponto e
avaliar se será necessário a injeção, ou remoção, de carga (energia) ao 
sistema. A presença de acessórios em uma
tubulação é quase que inevitável, seja por necessidades de otimização de 
espaço, seja por segurança ou por qualquer
outro motivo que seja necessária a instalação de uma válvula, por exemplo. 
Portanto, a investigação da perda de carga em
um escoamento é vital ao desenvolvimento de um projeto que atenda, com 
qualidade e segurança, o objetivo do sistema.

%%% Local Variables:
%%% mode: latex
%%% TeX-master: "../main_archive"
%%% End:
