\chapter{Resultados e Discussão}
\label{chap:resultados}

\section{Cálculo da vazão}
\label{sec:vazao}

Conforme procedimento experimental, apresenta-se a tabela \ref{t:tempos}, a
qual relaciona para cada amostragem os tempos de preenchimento dos volumes
analisados pela razão volumétrica do fluido de trabalho -- água.

\begin{table}[H]
\centering
\caption{Tempos amostrais de preenchimento.}
\label{t:tempos}
\begin{tabular}{|c|c|c|c|c|}
\hline
\multicolumn{5}{|c|}{\textbf{Tempo para preenchimento (s)}}                                      \\ \hline
\multicolumn{1}{|l|}{} & \multicolumn{2}{c|}{\textbf{0$^\circ$}}    & \multicolumn{2}{c|}{\textbf{45$^\circ$}}   \\ \hline
\textbf{Amostra}       & \textbf{8L} & \textbf{12L} & \textbf{8L} & \textbf{12L} \\ \hline
1                      & 38,780          & 55,500           & 84,455          & 126,570          \\ \hline
2                      & 47,570          & 71,000           & 85,290          & 127,245          \\ \hline
3                      & 46,140          & 70,520           & 85,980          & 127,050          \\ \hline
4                      & 47,185          & 71,155           & 85,175          & 127,205          \\ \hline
\end{tabular}
\end{table}

As vazões obtidas para cada amostragem foram calculadas considerando os tempos
necessários para que o volume ocupado pelo fluido no recipiente refratário
utilizado alcançasse 8 e 12L, utilizando a seguinte equação:

\newcommand{\Vol}{\mathop{\ooalign{\hfil$V$\hfil\cr\kern0.08em--\hfil\cr}}\nolimits}
\begin{equation}
  Q = \frac{\Vol}{\Delta t}
\end{equation}

onde $Q$ é a vazão e $\Vol$ é o volume preenchido durante o intervalo de tempo
$\Delta t$. Assim, utilizando-se dos dados constantes na tabela \ref{t:tempos},
apresenta-se a tabela seguinte, a qual relaciona a vazão do sistema para cada
amostra.

\begin{table}[H]
\centering
\caption{Vazão amostral por ângulo de abertura.}
\label{t:vazao}
\begin{tabular}{|c|c|c|c|c|c|c|}
\hline
\multicolumn{7}{|c|}{\textbf{Vazão ($m^3/s$)}}                                                               \\ \hline
\textbf{}        & \multicolumn{3}{c|}{\textbf{$0^\circ$}}     & \multicolumn{3}{c|}{\textbf{$45^\circ$}}    \\ \hline
\textbf{Amostra} & \textbf{8L} & \textbf{12L} & \textbf{Média} & \textbf{8L} & \textbf{12L} & \textbf{Média} \\ \hline
1                & 2,06292E-04 & 2,16216E-04  & 2,11254E-04    & 9,47250E-05 & 9,48092E-05  & 9,47671E-05    \\ \hline
2                & 1,68173E-04 & 1,69014E-04  & 1,68594E-04    & 9,37976E-05 & 9,43063E-05  & 9,40519E-05    \\ \hline
3                & 1,73385E-04 & 1,70164E-04  & 1,71775E-04    & 9,30449E-05 & 9,44510E-05  & 9,37479E-05    \\ \hline
4                & 1,69545E-04 & 1,68646E-04  & 1,69090E-04    & 9,39243E-05 & 9,43359E-04  & 9,41301E-05    \\ \hline
5                & 1,68315E-04 & 1,69803E-04  & 1,69059E-04    & 9,48148E-05 & 9,42840E-04  & 9,45494E-04    \\ \hline
\end{tabular}
\end{table}

\section{Cálculo da velocidade do fluido}
\label{sec:VelFluido}

A partir dos dados da tabela \ref{t:vazao}, calculou-se as respectivas
velocidades do fluido no sistema para cada amostra utilizando a equação

\begin{equation}
  v = \frac{4 Q}{\pi D^2}
\end{equation}

onde $v$ é a velocidade média do fluido e $D$ é o diâmetro hidráulico da
tubulação.

Tais velocidades foram consideradas constantes em todo o sistema, pois o fluido
de trabalho foi a água no estado líquido, portanto incompressível e a área da
seção transversal do escoamento interno manteve-se constante (tubo rígido). Tais
velocidades estão relacionadas às respectivas amostragens, conforme a seguinte
tabela.

\begin{table}[H]
\centering
\caption{Velocidade e número de Reynolds por ângulo de abertura.}
\label{t:Reynolds}
\begin{tabular}{|c|c|c|c|c|}
\hline
\multicolumn{5}{|c|}{\textbf{Velocidade e  número de Reynolds}}                                                      \\ \hline
\textbf{}        & \multicolumn{2}{c|}{\textbf{$0^\circ$}}         & \multicolumn{2}{c|}{\textbf{$45^\circ$}}        \\ \hline
\textbf{Amostra} & \textbf{Velocidade ($m/s$)} & \textbf{Reynolds} & \textbf{Velocidade ($m/s$)} & \textbf{Reynolds} \\ \hline
1                & 0,67244256                  & 13397,8475        & 0,30165304                  & 6010,18103        \\ \hline
2                & 0,53665026                  & 10692,3012        & 0,29937664                  & 5964,82561        \\ \hline
3                & 0,54677655                  & 10894,0590        & 0,29840899                  & 5945,54597        \\ \hline
4                & 0,53824820                  & 10724,1389        & 0,29962540                  & 5969,78182        \\ \hline
5                & 0,53813161                  & 10721,8157        & 0,30096016                  & 5996,37588        \\ \hline
\end{tabular}
\end{table}

\section{Cálculo do número de Reynolds (Re)}
\label{sec:Reynolds}

O $Re$ do sistema foi calculado para cada amostragem considerando a equação
\eqref{numReynolds}, na qual a variável geométrica característica para tubos
circulares é o diâmetro. As propriedades do fluido foram consultadas em
\citeay{Geankoplis93} considerando a pressão local igual a 1 atm ($101,325$ kPa) e
a $20^\circ$ C ($293,15$ K). Tais valores são apresentados na tabela
\ref{t:Reynolds}.

Conforme a tabela \ref{t:Reynolds}, cada amostragem apresentou $Re > 2000$ (valor
de referência utilizado conforme descrito no referencial teórico) caracterizando
para cada amostragem escoamento em regime turbulento.


\section{Cálculo do coeficiente de perda localizada ($k_f$)}
\label{sec:carga}

Conforme \citeay{Cremasco14}, em regime turbulento o coeficiente de perda de
carga localizada ($k_f$) da equação \eqref{PerdaDeCarga} é constante. Portanto,
tal coeficiente pode ser avaliado ao ser isolado da equação
\eqref{PerdaDeCarga}, uma vez que a perda de carga total referente à
configuração do sistema avaliado pode ser determinada e igualada a perda de
carga localizada, pois as perdas de cargas maiores foram desprezadas em
decorrência da pequena distância entre os pontos analisados e da rugosidade
relativa do tubo (de PVC) ser desprezível (tubo liso).

Apresentam-se, então, as seguintes tabelas, cujas quais relacionam a amostragem
às respectivas alturas de pressão (Tabela \ref{t:alturas}) e os coeficientes
de perda de carga localizados calculados (Tabela \ref{t:kf}). Para cálculo dos
valores de perda de carga e do coeficiente de perda de carga, as seguintes
equações foram utilizadas:

\begin{equation}\label{eq:hL}
h_L = z_1 - z_2
\end{equation}

\begin{equation}\label{eq:kf}
  k_f = \frac{2 g h_L}{v^2}
\end{equation}

onde $z_1$ e $z_2$ são a altura da coluna de água coletada nos manômetros
situados antes e após a válvula esférica, respectivamente. A equação
\eqref{eq:hL} pode ser obtida através da equação \eqref{eqBern}, aplicando-a
corretamente para o sistema apresentado.

\begin{table}[H]
\centering
\caption{Altura manométrica amostral por ângulo de abertura.}
\label{t:alturas}
\begin{tabular}{|c|c|c|c|c|}
\hline
\multicolumn{5}{|c|}{\textbf{Alturas (cm)}}                                                           \\ \hline
\textbf{}        & \multicolumn{2}{c|}{\textbf{$0^\circ$}} & \multicolumn{2}{c|}{\textbf{$45^\circ$}} \\ \hline
\textbf{Amostra} & \textbf{z1 (cm)}   & \textbf{z2 (cm)}   & \textbf{z1 (cm)}    & \textbf{z2 (cm)}   \\ \hline
1                & 9,0                & 3,5                & 71,6                & 3,3                \\ \hline
2                & 7,6                & 3,6                & 72,2                & 3,1                \\ \hline
3                & 7,4                & 3,7                & 72,3                & 3,4                \\ \hline
4                & 7,5                & 4,0                & 72,5                & 3,6                \\ \hline
5                & 7,8                & 4,2                & 71,9                & 3,5                \\ \hline
\end{tabular}
\end{table}

\begin{table}[H]
\centering
\caption{Perda de carga e coeficiente de perda de carga amostral por ângulo de abertura.}
\label{t:kf}
\begin{tabular}{|c|c|c|c|c|}
\hline
\multicolumn{5}{|c|}{\textbf{Perda de Carga e Coeficiente de Perda de Carga}}                               \\ \hline
\textbf{}              & \multicolumn{2}{c|}{\textbf{$0^\circ$}} & \multicolumn{2}{c|}{\textbf{$45^\circ$}} \\ \hline
\textbf{Amostra}       & \textbf{$h_L$ (m)}   & \textbf{$k_f$}   & \textbf{$h_L$ (m)}    & \textbf{$k_f$}   \\ \hline
1                      & 0,055                & 2,38644435       & 0,683                 & 147,266607       \\ \hline
2                      & 0,040                & 2,72506149       & 0,691                 & 151,265967       \\ \hline
3                      & 0,037                & 2,42818046       & 0,689                 & 151,807917       \\ \hline
4                      & 0,035                & 2,37029210       & 0,689                 & 150,577813       \\ \hline
5                      & 0,036                & 2,43907135       & 0,684                 & 148,162087       \\ \hline
\textbf{Desvio Padrão} & 0,00826438           & 0,14550331       & 0,00349285            & 1,99282035       \\ \hline
\textbf{Média}         & 0,0406               & 2,52190594       & 0,6872                & 149,812607       \\ \hline
\end{tabular}
\end{table}

A NBR 12214/1992 (Projeto de sistema de bombeamento de água para abastecimento
público, \cite{NBR12214}), em seu anexo B, traz valores para os coeficientes de
perda de carga localizada para a equação \eqref{PerdaDeCarga} para o acessório
utilizado -- válvula esférica. Para a abertura de $0^\circ$ (totalmente aberta),
$k_f$ = 0. Já para abertura de $45^\circ$, $k_f$ = $31,2$.

Conforme a tabela acima, para a abertura de $0^\circ$ o $k_f$ obtido foi igual a
$2,5219 \pm 0,1455$. Já para a abertura de $45^\circ$, o $k_f$ obtido foi igual
a $149,8126 \pm 1,9928$. Ambos os resultados diferiram, estatisticamente ao
nível de 95\% de confiança, dos respectivos valores referenciados pela norma
supracitada.

Tais diferenças podem ser justificadas devido às adaptações realizadas no
sistema analisado, que podem ter prejudicado os resultados para as duas
configurações. Além disso, há que se considerar os erros sistemáticos e
aleatórios presentes durante a aquisição dos dados.

Dos erros sistemáticos, citam-se:
\begin{itemize}
\item Instrumental: Na determinação do volume do refratário inferior, o qual
  pode ter sido mal calibrado;
\item Observacional: Erro paralaxe na leitura dos tempos;
\item Teórico: Oriundo da simplificação das equações que modelam o sistema.
\end{itemize}

Dos erros aleatórios, cita-se:
\begin{itemize}
\item Observacional: Erro na leitura da altura da coluna de fluido.
\end{itemize}

%%% Local Variables:
%%% mode: latex
%%% TeX-master: "../main_archive"
%%% End:
