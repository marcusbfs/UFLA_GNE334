\chapter{Resultados e Discussão}
\label{chap:resultados}

\section{Cálculo da vazão}
\label{sec:vazao}

Conforme procedimento experimental, apresenta-se a tabela \ref{t:tempos}, cuja
qual relaciona para cada amostragem os tempos de preenchimento dos volumes
analisados pela tazão volumétrica do fluido de trabalho -- água.

\begin{table}[H]
\centering
\caption{Tempos amostrais de preenchimento.}
\label{t:tempos}
\begin{tabular}{|c|c|c|c|c|}
\hline
\multicolumn{5}{|c|}{\textbf{Tempo para preenchimento (s)}}                                      \\ \hline
\multicolumn{1}{|l|}{} & \multicolumn{2}{c|}{\textbf{0$^\circ$}}    & \multicolumn{2}{c|}{\textbf{45$^\circ$}}   \\ \hline
\textbf{Amostra}       & \textbf{8L} & \textbf{12L} & \textbf{8L} & \textbf{12L} \\ \hline
1                      & 38,780          & 55,500           & 84,455          & 126,570          \\ \hline
2                      & 47,570          & 71,000           & 85,290          & 127,245          \\ \hline
3                      & 46,140          & 70,520           & 85,980          & 127,050          \\ \hline
4                      & 47,185          & 71,155           & 85,175          & 127,205          \\ \hline
\end{tabular}
\end{table}

As vazões obtidas para cada amostragem foram calculadas considerando os tempos
necessários para que o volume ocupado pelo fluido no recipiente refratário
utilizado alcançasse 8 e 12L. Portanto, utilizando-se dos dados constantes na
tabela \ref{t:tempos}, apresenta-se a tabela seguinte, cuja qual relaciona a vazão do sistema
para cada amostra.

\begin{table}[H]
\centering
\caption{Vazão amostral por ângulo de abertura.}
\label{t:vazao}
\begin{tabular}{|c|c|c|c|c|c|c|}
\hline
\multicolumn{7}{|c|}{\textbf{Vazão ($m^3/s$)}}                                                               \\ \hline
\textbf{}        & \multicolumn{3}{c|}{\textbf{$0^\circ$}}     & \multicolumn{3}{c|}{\textbf{$45^\circ$}}    \\ \hline
\textbf{Amostra} & \textbf{8L} & \textbf{12L} & \textbf{Média} & \textbf{8L} & \textbf{12L} & \textbf{Média} \\ \hline
1                & 2,06292E-04 & 2,16216E-04  & 2,11254E-04    & 9,47250E-05 & 9,48092E-05  & 9,47671E-05    \\ \hline
2                & 1,68173E-04 & 1,69014E-04  & 1,68594E-04    & 9,37976E-05 & 9,43063E-05  & 9,40519E-05    \\ \hline
3                & 1,73385E-04 & 1,70164E-04  & 1,71775E-04    & 9,30449E-05 & 9,44510E-05  & 9,37479E-05    \\ \hline
4                & 1,69545E-04 & 1,68646E-04  & 1,69090E-04    & 9,39243E-05 & 9,43359E-04  & 9,41301E-05    \\ \hline
5                & 1,68315E-04 & 1,69803E-04  & 1,69059E-04    & 9,48148E-05 & 9,42840E-04  & 9,45494E-04    \\ \hline
\end{tabular}
\end{table}

\section{Cálculo da velocidade do fluido}
\label{sec:VelFluido}

A partir dos dados da tabela \ref{t:vazao}, calculou-se as respectivas
velocidades do fluido no sistema para cada amostra. Tais velocidades foram
consideradas constantes em todo o sistema, pois o fluido de trabalho foi a água
no estado líquido, portanto incompressível e a área da seção transversal do
escoamento interno manteve-se constante (tubo rígido). Tais velocidades estão
relacionadas às respectivas amostragens, conforme a seguinte tabela.

%%% Local Variables:
%%% mode: latex
%%% TeX-master: "../main_archive"
%%% End:
