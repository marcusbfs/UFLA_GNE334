\chapter{Procedimento Experimental}
\label{chap:procedimento}

% Inicialmente, com a ajuda de um profissional, furou-se um balde de 20L em dois
% lugares: um localizado próximo à borda, para o ladrão, e outro próximo à base,
% para permitir a entrada de água na tubulação.

%% ===================================================

Inicialmente, antes das medições, o fluxo de água da torneira foi controlado de
modo que o regime de escoamento se tornasse estacionário, ou seja, não houvesse
variação no tempo. Após isso, foram feitas várias medições para cada abertura da
válvula esférica, sendo elas: totalmente aberta e uma abertura de 45 graus.
Essas aberturas foram calculadas com o auxílio de um transferidor, acoplado à
válvula. 

Para realização do experimento, foram realizadas, simultâneamente, a medição da
vazão e da diferença de pressão causada pela válvula. Para a vazão, foram
demarcados em um balde de 15L níveis correspondentes a volumes de 8 e 12 litros;
também foi coletado o tempo necessário para que a água que deixa a tubulação
preencha os respectivos volumes, com auxílio de cronômetros.  Com o intuito de
minimizar erros, foram utilizados dois cronômetros e, posterioremente,
calculou-se a média aritmética dos tempos obtidos. Já para aquisição dos dados
de diferença de pressão, mediu-se, com trenas, as alturas de coluna de água em
cada uma das mangueiras.

O procedimento supracitado foi repetido cinco vezes, para cada abertura da
válvula, afim de garantir maior precisão nos resultados. Os dados obtidos
serão tratados no capítulo \ref{chap:resultados}.


%%% Local Variables:
%%% mode: latex
%%% TeX-master: "../main_archive"
%%% End:
