\chapter{Descrição do Projeto}\label{descricao}

\section{Materiais Utilizados}

 \begin{table}[H]\label{t:custos}
     \centering
     \begin{tabular}{l c }
         \toprule
         \textbf{Material} & \textbf{Valor (R\$)} \\
         \midrule
         Adaptador (conexão entre balde e cano) & 1,00 - 5,00 \\
         Baldes & 20,00 - 30,00 \\
         Canos PVC & 25,00 - 50,00 \\
         Cola Durepoxi & 10,00 - 20,00 \\
         Cotovelos 90$^\circ$ & 1,00 - 5,00\\
         Cronômetro & 10,00 - 20,00 \\
         Registro rosqueável & 30,00 - 50,00 \\
         Suportes para a estrutura (madeira) & 20,00 - 50,00 \\
         Trena & 10,00 - 15,00 \\
         Tubos transparentes (manômetro) & 15,00 - 20,00 \\
         Válvula esférica & 30,00 - 50,00 \\
         \midrule
         \textbf{TOTAL} & 172,00 - 315,00\\
         \bottomrule
     \end{tabular}
     \caption{Custos}
 \end{table}

\section{Montagem}

A estrutura se inicia com um recipiente posicionado a uma elevação 
considerável do nível convencionado como $z = 0$. Esse
recipiente é um balde de 20 litros, o qual possui um ``ladrão''
na sua cota máxima, afim de evitar-se transbordamento do fluido excedente e também
está ligado, em sua parte inferior, 
à tubulação de PVC através de um
adaptador, e após essa conexão foi instalado um registro para controle da 
saída do fluido. Os tubos de PVC possuem diâmetro de 20 milímetros, então todos
os acessórios são correspondentes a esse diâmetro. A seguir, a tubulação faz
uma curva de $90^\circ$, descendo até o nível $z = 0$, 
onde faz uma nova curva de $90^\circ$
e prossegue reta até o local onde é
acoplada uma válvula esférica. Para a análise da perda de carga nessa válvula, é
necessário medir a diferença de pressão antes e depois dela, com o auxílio de um
manômetro.

Manômetros são instrumentos utilizados para determinação da diferença de 
pressão entre dois pontos de um fluido através
da diferença de elevação entre estes dois pontos. Dessa forma, o manômetro 
é constituído de dois tubos transparentes
acoplados ao tubo de PVC, um deles um pouco antes da válvula e o outro um 
pouco depois. Nesses tubos serão medidas as colunas do fluido, aqui, a água.

Em seguida, a estrutura termina com a tubulação despejando o fluido em um 
outro recipiente de 15 litros e a medição da vazão será feita cronometrando-se o
tempo necessário para que se alcance um volume conhecido desse balde.

\section{Dificuldades Encontradas}%
\label{sec:dificuldades}
A ESCREVER.

\section{Adaptações Realizadas em Relação ao Projeto Proposto}%
\label{sec:dificuldades}
A ESCREVER.

%%% Local Variables:
%%% mode: latex
%%% TeX-master: "../main_archive"
%%% End:
