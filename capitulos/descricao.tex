\chapter{Descrição do Projeto}\label{descricao}

\section{Materiais Utilizados}

 \begin{table}[H]\label{t:custos}
     \centering
     \begin{tabular}{l c }
         \toprule
         \textbf{Material} & \textbf{Valor (R\$)} \\
         \midrule
         Adaptador (conexão entre balde e cano) & 1,00 - 5,00 \\
         Baldes & 20,00 - 30,00 \\
         Canos PVC & 25,00 - 50,00 \\
         Cola Durepoxi & 10,00 - 20,00 \\
         Cotovelos 90$^\circ$ & 1,00 - 5,00\\
         Cronômetro & 10,00 - 20,00 \\
         Registro rosqueável & 30,00 - 50,00 \\
         Suportes para a estrutura (madeira) & 20,00 - 50,00 \\
         Trena & 10,00 - 15,00 \\
         Tubos transparentes (manômetro) & 15,00 - 20,00 \\
         Válvula esférica & 30,00 - 50,00 \\
         \midrule
         \textbf{TOTAL} & 172,00 - 315,00\\
         \bottomrule
     \end{tabular}
     \caption{Custos}
 \end{table}

\section{Montagem}

A estrutura se inicia com um recipiente posicionado a uma elevação 
considerável do nível convencionado como $z = 0$. Esse
recipiente será um balde de alta capacidade volumétrica, o qual possuirá um ``ladrão''
na sua cota máxima, afim de evitar-se transbordamento do fluido excedente e também
estará ligado, em sua parte inferior, 
à tubulação de PVC através de um
adaptador, e após essa conexão será instalado um registro para controle da 
saída do fluido. A seguir, a tubulação fará
uma curva de $90^\circ$, descendo até o nível $z = 0$, 
onde fará uma nova curva de $90^\circ$
e prosseguirá reta até o local onde será
acoplado o acessório escolhido. Inicialmente, será analisada a perda de carga 
em uma válvula esférica e para tal é
necessário medir a diferença de pressão no tubo antes e depois da
válvula, com o auxílio de um manômetro.

Manômetros são instrumentos utilizados para determinação da diferença de 
pressão entre dois pontos de um fluido através
da diferença de elevação entre estes dois pontos. Dessa forma, o manômetro 
será constituído de dois tubos transparentes
acoplados ao tubo de PVC, um deles um pouco antes da válvula e o outro um 
pouco depois. Nesses tubos serão medidas as colunas de líquido. 

Em seguida, a estrutura irá terminar com a tubulação despejando o fluido em um 
outro recipiente de capacidade
volumétrica conhecida de forma que, através da cronometragem do tempo que esse 
recipiente leva para completar seu volume, seja possível 
determinar a vazão volumétrica. 

%%% Local Variables:
%%% mode: latex
%%% TeX-master: "../main_archive"
%%% End:
