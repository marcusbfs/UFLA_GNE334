\chapter{Descrição do Projeto}\label{descricao}

\section{Materiais Utilizados}

 \begin{table}[H]\label{t:custos}
     \centering
     \begin{tabular}{l r}
         \toprule
         \textbf{Material} & \textbf{Quantidade} \\
         \midrule
         Adaptador (conexão entre balde e cano) & 1 unidade \\
         Baldes & 2 unidades \\
         Canos PVC 25 mm &  3 metros \\
         Cola Durepoxi & 1 unidade \\
         Cola para PVC & 1 unidade \\
         Cotovelos 90$^\circ$ 25 mm & 2 unidades \\
         Cronômetro & 2 unidades \\
         Régua & 2 unidades \\
         Silicone & 1 unidade \\
         Tubos transparentes (manômetro) & 2 unidades \\
         Válvula esférica 20 mm & 2 unidades \\
         \bottomrule
     \end{tabular}
     \caption{Materiais utilizados}
 \end{table}

\section{Montagem}
\label{sec:montagem}

A estrutura se inicia com um recipiente posicionado a uma elevação 
considerável do nível convencionado como $z = 0$. Esse
recipiente é um balde de 20 litros, o qual possui um ``ladrão''
na sua cota máxima, afim de evitar-se transbordamento do fluido excedente e também
está ligado, em sua parte inferior, 
à tubulação de PVC através de um
adaptador, e após essa conexão foi instalado um registro para controle da 
saída do fluido. Os tubos de PVC possuem diâmetro de 20 milímetros, então todos
os acessórios são correspondentes a esse diâmetro. A seguir, a tubulação faz
uma curva de $90^\circ$, descendo até o nível $z = 0$, 
onde faz uma nova curva de $90^\circ$
e prossegue reta até o local onde é
acoplada uma válvula esférica. Para a análise da perda de carga nessa válvula, é
necessário medir a diferença de pressão antes e depois dela, com o auxílio de um
manômetro.

Manômetros são instrumentos utilizados para determinação da diferença de 
pressão entre dois pontos de um fluido através
da diferença de elevação entre estes dois pontos. Dessa forma, o manômetro 
é constituído de dois tubos transparentes
acoplados ao tubo de PVC, um deles um pouco antes da válvula e o outro um 
pouco depois. Nesses tubos serão medidas as colunas do fluido, aqui, a água.

Em seguida, a estrutura termina com a tubulação despejando o fluido em um 
outro recipiente de 15 litros e a medição da vazão será feita cronometrando-se o
tempo necessário para que se alcance um volume conhecido desse balde.

\section{Adaptações Realizadas em Relação ao Projeto Proposto}%
\label{sec:adaptacoes}

Uma vez que a estrutura universitária não conta com módulos didáticos
relacionados ao aprendizado prático da mecânica dos fluidos, houve a necessidade
da adaptação de um aparato experimental que viabilizasse o estudo e a coleta dos
dados necessários à avaliação do coeficiente de perda de carga localizada para
uma válvula esférica.

Das adaptações necessárias, citam-se:
\begin{itemize}
\item Utilização de um refratário superior com vazão secundária (ladrão) a fim
  de promover o estabelecimento do escoamento do regime estacionário;
\item Adaptação de um manômetro a fim de avaliar a perda de carga relacionada às
  diferenças de altura de pressão;
\item Utilização de um refratário graduado, também adaptado, para a coleta da
  descarga do sistema de tubulação para a avaliação da vazão do sistema;
\item Utilização de suportes a fim de igualar as alturas entre os pontos de
  análise;
\item Utilização de uma válvula esférica na parte superior, que serve apenas
  para controlar a entrada de água à tubulação. Uma vez que esse era o acessório
  de menor preço, foi utilizado nos dois locais.
\item Utilização de suporte total para o aparato experimental completo.
\end{itemize}

\section{Dificuldades Encontradas}%
\label{sec:dificuldades}

As dificuldades encontradas no percurso do presente projeto podem ser
relacionadas à própria montagem experimental, oriundas das necessárias
adaptações ao aparato, conforme descrito na seção \ref{sec:adaptacoes}, como
também durante a coleta de dados e, por fim, na disponibilidade de dados na
literatura para comparação dos valores encontrados.

Das dificuldades relacionadas à execução e coleta de dados, citam-se:
\begin{itemize}
\item A dificuldade no controle da altura do fluido no refratário (balde)
  superior para a garantia de regime estacionário;
\item Adaptações realizadas na instalação do manômetro, cujas paredes de seus
  tubos conectados à tubulação, ainda que, minimamente, para garantia da
  sustentação, vieram a interferir no escoamento (tal qual um acessório);
\item A dificuldade em garantir a diferença de altura nula entre os pontos de
  análise;
\item A dificuldade na vedação da junção tubo (manômetro) - tubulação,
acarretando em pequenas variações no tempo da altura do fluido, dificultando a
coleta de dados;
\item Local de trabalho pouco adequado.
\end{itemize}


%%% Local Variables:
%%% mode: latex
%%% TeX-master: "../main_archive"
%%% End:
