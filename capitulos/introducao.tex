\chapter{Introdução} \label{intro}

\section{Referencial Teórico}

Segundo \citeay{Munson04} ``a mecânica dos fluidos é a parte da mecânica aplicada que se dedica à
análise do comportamento dos líquidos e gases, tanto em equilíbrio quanto em movimento.
Obviamente, o escopo da mecânica dos fluidos abrange um vasto conjunto de problemas.''

Na análise do escoamento de fluidos é necessário o balanço de massa e de momento 
para desenvolvimento de equações apropriadas aos fenômenos a serem analisados \citeayp{Cremasco14}.
A descrição matemática da fluidodinâmica é dada pelas  equações de massa \eqref{eqcontinuidade}
e equações do movimento \eqref{eqQDM}.

\begin{equation}\label{eqcontinuidade}
    \frac{\partial\rho}{\partial t} + 
    \vec{\nabla} \cdot \rho \textbf{u}  = 0
\end{equation}

\begin{equation}\label{eqQDM}
    \rho \left(
        \frac{\partial \rho}{\partial t} + 
    \textbf{u} \cdot \vec{\nabla}\rho\textbf{u}
    \right)
    = 
    \rho\vec{\nabla}\varphi -
    \vec{\nabla} P +
    \vec{\nabla}\cdot \tau
\end{equation}

onde $\rho$ é a massa específica do fluido, \textbf{u} é a sua velocidade, $\vec{\nabla}\varphi$
é o vetor intensidade do campo, $\vec{\nabla} P$ é o gradiente de pressão e $\tau$ é a tensão extra,
dinâmica ou viscosa.

Ao manipular as equações \eqref{eqcontinuidade} e \eqref{eqQDM}, considerando que o fluido se
encontra em regime permanente, é invíscido e incompressível, chega-se a equação simplificada
para a energia mecânica:

\begin{equation}\label{eqBern}
    \left(
        z + \frac{P}{\rho g} + \frac{u^2}{2g}
    \right)_{1}
    =
    \left(
        z + \frac{P}{\rho g} + \frac{u^2}{2g}
    \right)_{2} + h_L
\end{equation}

onde os subíndices $1$ e $2$ indicam dois pontos de escoamento dinstintos, e $h_L$ é
a perda de carga dada em unidades de comprimento.

Conforme \citeay{Cremasco14}, um fluido em um sistema de escoamento que apresente acessórios
estará sujeito a uma perda de carga, uma vez que, segundo o autor, quando o fluido entra em contato
com um acessório ocorre a separação de uma camada do escoamento, formando correntes turbulentas
que dissipam energia mecânica. Essa energia dissipada representa a conversão irreversível de
energia mecânica em energia térmica não desejada e a perda de energia por transferência de 
calor \citeayp{Fox14}.

A perda de carga oriunda de um acessório em um sistema de escoamento é definida por meio da equação:

\begin{equation}\label{PerdaDeCarga}
    h_L = k_f \frac{u^2}{2g}
\end{equation}

onde $k_f$ é o coeficiente de perda de carga localizada característico do acessório, 
$u$ é a velocidade média do fluido e $g$ é a aceleração local da gravidade. 

Ainda conforme o autor, $k_f$ é constante apenas em regime turbulento. Podem ser 
encontradas na literatura tabelas que apresentam os valores de $k_f$ para válvulas
e acessórios específicos.

De acordo com \citeay{Potter15}, o regime turbulento é caracterizado pelo movimento
aleatório das partículas do fluido na trajetória do escoamento. Tal regime pode ser previsto
por um número adimensional denominado por número de Reynolds, o qual é calculado, para tubulações
cilíndricas, conforme:

\begin{equation}\label{numReynolds}
    Re = \frac{u D}{\nu}
\end{equation}
 
onde $D$ é o diâmetro característico da tubulação e $\nu$ é a viscosidade cinemática do fluido.

Ainda conforme \citeay{Potter15}, valores para o número de Reynolds acima de 2000 são 
características de um escoamento em regime turbulento, sendo este o valor mínimo
usado na maioria das aplicações em engenharia. Entretanto, tal valor pode variar
em decorrência da geometria e da rugosidade das paredes da tubulação.

Portanto, além do cálculo adimensional, pode ser realizada uma análise 
qualitativa por meio do despejo de partículas corantes na corrente de escoamento,
avaliando se ocorre uma mistura imediata, ou não. Caso não, o regime é laminar.

\section{Aplicabilidade}

O cálculo da perda de carga é necessário em qualquer tipo de planejamento
e instalação de tubulações, uma vez que o mesmo está relacionado diretamente
com o matérial utilizado e as características da tubulação.

Algumas situações que requerem uma análise da perda de carga podem ser 
exemplificadas como: possível cavitação em bombas, danos à tubulação
de edifícios, insuficiência na captação de água e/ou esgoto para
tratamento e controle de fluxo em reatores e outros  equipamentos da indústria química.

\section{Objetivos}

O objetivo desse projeto é determinar a perda de carga e o coeficiente de 
perda de carga localizada em um tubo conectado a dois reservatórios com
um fluxo de água constante através de acessórios (individualmente) como
registros e cotovelos a partir da variação de pressão entre os pontos
anterior e posterior aos acessórios.

%%% Local Variables:
%%% mode: latex
%%% TeX-master: "../main_archive"
%%% End:
